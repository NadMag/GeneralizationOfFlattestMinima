%% LyX 2.3.6.1 created this file.  For more info, see http://www.lyx.org/.
%% Do not edit unless you really know what you are doing.
\documentclass[english]{article}
\PassOptionsToPackage{natbib=true}{biblatex}
\usepackage[T1]{fontenc}
\usepackage[latin9]{inputenc}
\usepackage{geometry}
\geometry{verbose,tmargin=2cm,bmargin=2cm,lmargin=2cm,rmargin=2cm}
\setlength{\parindent}{0bp}
\usepackage{xcolor}
\usepackage{babel}
\usepackage{refstyle}
\usepackage{enumitem}
\usepackage{amsmath}
\usepackage{amsthm}
\usepackage{amssymb}
\usepackage[unicode=true,pdfusetitle,
 bookmarks=true,bookmarksnumbered=false,bookmarksopen=false,
 breaklinks=false,pdfborder={0 0 0},pdfborderstyle={},backref=false,colorlinks=false]
 {hyperref}

\makeatletter

%%%%%%%%%%%%%%%%%%%%%%%%%%%%%% LyX specific LaTeX commands.

\AtBeginDocument{\providecommand\propref[1]{\ref{prop:#1}}}
\AtBeginDocument{\providecommand\thmref[1]{\ref{thm:#1}}}
\RS@ifundefined{subsecref}
  {\newref{subsec}{name = \RSsectxt}}
  {}
\RS@ifundefined{thmref}
  {\def\RSthmtxt{theorem~}\newref{thm}{name = \RSthmtxt}}
  {}
\RS@ifundefined{lemref}
  {\def\RSlemtxt{lemma~}\newref{lem}{name = \RSlemtxt}}
  {}


%%%%%%%%%%%%%%%%%%%%%%%%%%%%%% Textclass specific LaTeX commands.
\newlength{\lyxlabelwidth}      % auxiliary length 
\theoremstyle{definition}
\newtheorem{defn}{\protect\definitionname}
\theoremstyle{plain}
\newtheorem{prop}{\protect\propositionname}
\theoremstyle{plain}
\newtheorem{thm}{\protect\theoremname}
\theoremstyle{plain}
\newtheorem{fact}{\protect\factname}

%%%%%%%%%%%%%%%%%%%%%%%%%%%%%% User specified LaTeX commands.
\usepackage{amsmath}
\usepackage{amsthm}
% Necessary Commands:
\usepackage{autobreak}
\usepackage{relsize}
\DeclareMathOperator*{\argmin}{argmin}
\DeclareMathOperator*{\argmax}{argmax}

%referencing layouts
\newref{prop}{refcmd={\emph{Proposition\,\ref{#1}}}}
\newref{thm}{refcmd={\emph{Theorem\,\ref{#1}}}}
%\providecommand{\propautorefname}{Proposition}

\providecommand\phantomsection{}

%Double struck 0, 1
\usepackage{bbold}

\makeatother

\usepackage[bibstyle=verbose,citestyle=authoryear,backend=bibtex8]{biblatex}
\providecommand{\definitionname}{Definition}
\providecommand{\factname}{Fact}
\providecommand{\propositionname}{Proposition}
\providecommand{\theoremname}{Theorem}

\addbibresource{main.bib}
\begin{document}
\global\long\def\kalivectormultiplication#1#2#3#4#5#6{\left[\begin{matrix}#1\\
 #2\\
 #3 
\end{matrix}\right]\times\left[\begin{matrix}#4\\
 #5\\
 #6 
\end{matrix}\right]=\left[\begin{matrix}\left(#2\right)\cdot\left(#6\right)-\left(#3\right)\cdot\left(#5\right)\\
 \left(#3\right)\cdot\left(#4\right)-\left(#1\right)\cdot\left(#6\right)\\
 \left(#1\right)\cdot\left(#5\right)-\left(#2\right)\cdot\left(#4\right) 
\end{matrix}\right]}%

\global\long\def\kaliseries#1#2#3{#1_{1}#2#1_{2}#2\ldots#2#1_{#3}}%

\global\long\def\kaliseriesZero#1#2#3{#1_{0}#2#1_{1}#2\ldots#2#1_{#3}}%

\global\long\def\kalidoubleseries#1#2#3#4#5{#1_{1}#2#3_{1}#4#1_{2}#2#3_{2}#4\ldots#1_{#5}#2#3_{#5}}%

\global\long\def\kalidoubleseriesZero#1#2#3#4#5{#1_{0}#2#3_{0}#4#1_{1}#2#3_{1}#4\ldots#1_{#5}#2#3_{#5}}%

\global\long\def\kaliOverset#1#2{\overset{#2}{#1}}%

\global\long\def\kaliUnderset#1#2{\underset{#2}{#1}}%

\global\long\def\an#1{\left(#1\right)_{n=1}^{\infty}}%

\global\long\def\kalim#1{\lim_{n\rightarrow\infty}\left(#1\right)}%

\global\long\def\flim#1#2{\lim_{x\rightarrow#2}\left(#1\right)}%

\global\long\def\kaliCupDot{\mathbin{\cupdot}}%

\global\long\def\kaliBigCupDot{\mathbin{\bigcupdot}}%

\global\long\def\kaliNabla{\overline{\nabla}}%

\global\long\def\kaliTauEquiv{\equiv_{\text{tau}}}%

\global\long\def\kaliV{\textcolor{green}{\checkmark}}%

\global\long\def\kaliX{{\color{red}\boldsymbol{\times}}}%

\global\long\def\kaliFourier{\mathcal{F}}%

\global\long\def\kalaplace{\mathcal{L}}%

\global\long\def\kaliHemiltonian{\mathcal{H}}%

\global\long\def\kaliPowerSet{\mathcal{P}}%

\global\long\def\kaliABasis{\mathcal{A}}%

\global\long\def\kaliBBasis{\mathcal{B}}%

\global\long\def\kaliOBasis{\mathcal{O}}%

\global\long\def\kaliCBasis{\mathcal{C}}%

\global\long\def\kaliMBasis{\mathcal{M}}%

\global\long\def\kaliNBasis{\mathcal{N}}%

\global\long\def\kaliFancyF{\mathscr{F}}%

\global\long\def\kaliFancyB{\mathscr{B}}%

\global\long\def\kaliForall{\mathbin{\forall}}%

\global\long\def\kaliExists{\mathbin{\exists}}%

\global\long\def\kaliField{\mathbb{F}}%

\global\long\def\kaliComplex{\mathbb{C}}%

\global\long\def\kaliReal{\mathbb{R}}%

\global\long\def\kaliRational{\mathbb{Q}}%

\global\long\def\kaliNatural{\mathbb{N}}%

\global\long\def\kaliInteger{\mathbb{Z}}%

\global\long\def\kaliProbability{\mathbb{P}}%

\global\long\def\kaliExpectedValue{\mathbb{E}}%

\global\long\def\kaliUnitOperator{\mathbb{I}}%

\global\long\def\kaliHbb{\mathbb{H}}%

\global\long\def\kaliGothica{\mathfrak{a}}%

\global\long\def\kaliGothicb{\mathfrak{b}}%

\global\long\def\kaliGothicc{\mathfrak{c}}%

\global\long\def\kaliRank{\text{Rank}}%

\global\long\def\kaliIm{\text{Im}}%

\global\long\def\kaliSpan{\text{Span}}%

\global\long\def\kaliSg{\text{sg}}%

\global\long\def\kaliOrd{\text{Ord}}%

\global\long\def\kaliOtp{\text{otp}}%

\global\long\def\kaliWo{\text{WO}}%

\global\long\def\kaliCard{\text{Card}}%

\global\long\def\kaliCf{\text{cf}}%

\global\long\def\kaliSupp{\text{Supp}}%

\global\long\def\kaliGeo{\text{Geo}}%

\global\long\def\kaliBin{\text{Bin}}%

\global\long\def\kaliBer{\text{Ber}}%

\global\long\def\kaliPoi{\text{Poi}}%

\global\long\def\kaliExp{\text{Exp}}%

\global\long\def\kaliCov{\text{Cov}}%

\global\long\def\kaliVar{\text{Var}}%

\global\long\def\kaliSinc{\text{sinc}}%

\global\long\def\kaliErf{\text{erf}}%

\global\long\def\kaliDom{\text{dom}}%

\global\long\def\kaliRange{\text{range}}%

\global\long\def\kaliId{\text{Id}}%

\global\long\def\kaliSeq{\text{Seq}}%

\global\long\def\kaliAlg{\text{Alg}}%

\global\long\def\kaliTrace{\text{Tr}}%

\global\long\def\kaliSent{\text{sent}}%

\global\long\def\kaliConst{\text{Const}}%

\global\long\def\kaliConstTerm{\text{constterm}}%

\global\long\def\kaliRel{\text{Rel}}%

\global\long\def\kaliFunc{\text{Func}}%

\global\long\def\kaliTerm{\text{term}}%

\global\long\def\kaliAtom{\text{atom}}%

\global\long\def\kaliForm{\text{form}}%

\global\long\def\kaliAssign{\text{assign}}%

\global\long\def\kaliInd{\text{ind}}%

\global\long\def\kaliTh{\text{Th}}%

\global\long\def\kaliAxLog{\text{AxLog}}%
 

\global\long\def\kaliConclusion{\text{cl}}%

\global\long\def\kaliRes{\text{Res}}%

\global\long\def\kaliGL{\text{GL}}%

\global\long\def\kaliSL{\text{SL}}%

\global\long\def\kaliLcm{\text{lcm}}%

\global\long\def\kaliAut{\text{Aut}}%

\global\long\def\kaliSyl{\text{Syl}}%

\global\long\def\kaliChar{\mathop{\text{char}}}%

\global\long\def\kaliTor{\text{tor}}%

\global\long\def\argmax{\operatornamewithlimits{argmax}}%

\global\long\def\argmin{\operatornamewithlimits{argmin}}%

\global\long\def\max{\operatornamewithlimits{max}}%

\global\long\def\min{\operatornamewithlimits{min}}%



\date{}
\author{Nadav Magar, Artion Makhlin, Noam Razin}
\title{On the Generalization of Flat Minima in Linear Neural Networks}
\maketitle

\section*{Problem Setting}

Consider the case of linear regression: given a sample $S=\left(x^{(i)},y^{(i)}\right)_{i=1}^{N}\subseteq\left(\mathbb{R}^{d_{x}}\times\mathbb{R}^{d_{y}}\right),S\stackrel{i.i.d}{\sim}{\scriptstyle D^{N}}$. 

Denote $X=\begin{pmatrix}\vert & \vert &  & \vert\\
\left[x^{(1)}\right] & \left[x^{(2)}\right] & ... & [x^{(N)}]\\
\vert & \vert &  & \vert
\end{pmatrix}\in\mathbb{R}^{d_{x}\times N},Y=\begin{pmatrix}\vert & \vert &  & \vert\\
\left[y^{(1)}\right] & \left[y^{(2)}\right] & ... & [y^{(N)}]\\
\vert & \vert &  & \vert
\end{pmatrix}\in\mathbb{R}^{d_{y}\times N}$, our objective is to learn a linear transformation $f:\mathbb{R}^{d_{x}}\rightarrow\mathbb{R}^{d_{y}}$
corresponding to a matrix $W\in\mathbb{R}^{d_{y}\times d_{x}}$, that
minimizes the empirical (quadratic) loss: $L(W)=\left\Vert WX-Y\right\Vert ^{2}$.
We study the overparameterized regime, in which $W$ is decomposed
as a product of matrices (Linear Neural Network): $W_{i}\in\mathbb{R}^{d_{i}\times d_{i-1}},i\in\left[m\right]$
where $d_{0}:=d_{x},d_{N}:=d_{y}$. 

Denote $W_{1:m}:=\prod_{i=1}^{m}W_{i}$ ,the overparameterized objective
is defined by: 
\begin{equation}
\phi(W_{1},W_{2},...,W_{m})=L(W_{1:m})\label{eq:overparamloss}
\end{equation}

We restrict our attention to the Realizable case, i.e. we there exists
a matrix $W^{\ast}\in\mathbb{R}^{d_{y}\times d_{x}}$ such that $L(W^{\ast})=0$.
Thus the set of global minima of $\phi$ is given by:
\begin{equation}
\Omega=\left\{ (W_{1},W_{2},...,W_{m})\in\prod_{i=1}^{m}\mathbb{R}^{d_{i}\times d_{i-1}}\ \Bigm\vert\ L(W_{1:m})=0\right\} 
\end{equation}
\citep{Mulayoff2020Unique} characterized the sharpness of a solution
by its Hessian's maximal eigenvalue. 
\begin{defn}[Minima Flatness]
Denote the vectorization of the network's parameters $(W_{1},W_{2},...,W_{m})$
by $w$, and the Hessian of $\phi$ at the corresponding point by
$H_{w}$, then the set of \emph{flattest }global minima is defined
to be:
\begin{equation}
\Omega_{0}=\argmin_{w\in\Omega}\lambda_{\mathrm{\max}}\left(H_{w}\right)\label{eq:flattestmin}
\end{equation}

Where in (\ref{eq:flattestmin}) we have used a slight abuse of notation
by identifying $(W_{1},W_{2},...,W_{m})$ with $w$ for brevity.
\end{defn}
Mulayoff et al. (2020) studied the case where a single end-to-end
solution exists:
\begin{enumerate}[label=\bfseries{A3}]
\item \label{assumption:nonsingular} The data's ambient dimension $d_{x}$
is smaller then the number of training samples $N$.
\end{enumerate}
In this case the unique solution can be written as $f^{\ast}(x)=Tx$
where: 
\begin{equation}
T=\hat{\Sigma}_{yx}\hat{\Sigma}_{x}^{-1}:=YX^{T}\left(XX^{T}\right)^{-1}
\end{equation}

We will deonte the the SVD of $T$ by:
\begin{equation}
T=USV^{T}
\end{equation}

\newpage{}

\section*{Balancedness of Flattest Solutions}

Theoretical study of optimization in deep learning is a highly active
research area. A recent approach who has seen significant results,
studies the trajectories of specific optimizers emanating from specific
initializations. Arrora et al. (2018) derived closed term expressions
for the dynamics induced by gradient-flow on LNN as defined in (\ref{eq:overparamloss}).
Using their analysis Arrora et al. (2019) have showed a convergence
result for both GF and GD. The authors propose the following definition:
\begin{defn}[Balancedness]
For $\delta\ge0$ we say that the weight matrices $\kaliseries W,m$
are $\delta-$balanced if:
\begin{equation}
\left\Vert W_{j+1}^{T}W_{j+1}-W_{j}W_{j}^{T}\right\Vert _{F}\le\delta\quad,\forall j\in[m-1]
\end{equation}

If $\kaliseries W,N$ are 0-balanced we simply say that they are balanced.
Let us denote the set of balanced weight matrices by:
\begin{equation}
\mathcal{B}=\left\{ (W_{1},W_{2},...,W_{m})\in\prod_{j=1}^{m}\mathbb{R}^{d_{i}\times d_{i-1}}\ \Bigm\vert\ W_{j+1}^{T}W_{j+1}=W_{j}W_{j}^{T}\quad,\forall j\in[m-1]\right\} 
\end{equation}

\medskip{}
\end{defn}
\begin{prop}[Arrora et al. ]
\label{prop:balancedness}Assume the weight matrices $\kaliseries W,N$
are initialized to be $\delta-\mathrm{balanced}$: 
\[
\left\Vert W_{j+1}^{T}(0)W_{j+1}(0)-W_{j}(0)W_{j}^{T}(0)\right\Vert _{F}\le\delta,\quad\forall j\in[m-1]
\]
 this property is conserved throughout optimization:
\[
\left\Vert W_{j+1}^{T}(t)W_{j+1}(t)-W_{j}(t)W_{j}^{T}(t),\quad j\right\Vert _{F}\le\delta\quad,\kaliForall j\in[m-1],\ t\ge0
\]
\end{prop}
\bigskip{}

Mulayoff et al. (2020) have showed that under the assumptions:
\begin{enumerate}[label=\bfseries{A\arabic{enumi}}]
\item  \label{assumption:capacity}The network has the capacity to implement
any linear function from $\mathbb{R}^{d_{x}}$ to $\mathbb{R}^{d_{y}}$:
$\min_{i\in\left[m\right]}\{d_{i}\}\ge\min\{d_{x},d_{y}\}.$
\item \label{assumption:white}The data is white, namely $\hat{\Sigma}_{x}=I_{d_{x}}$.
\end{enumerate}
The following sufficient condition for minima flatness holds:
\begin{thm}[Sufficient Condition, Mulayoff et al.]
\label{thm:sufficient}Assume (\ref{assumption:capacity}),(\ref{assumption:white}),(\ref{assumption:nonsingular}).
If a solution $w\in\Omega$ satisfies $\sigma_{\max}(W_{k})=\left(\sigma_{\max}\left(T\right)\right)^{\frac{1}{m}}$
for all $k\in[m]$, then necessarily $w\in\Omega_{0}.$
\end{thm}
Using this fact it can be shown that GF starting from a balanced initialization
can only converge to flattest solutions. By \propref{balancedness}
we can see that any balanced solution $w\in\Omega\cap\mathcal{B}$
is in fact a flattest minima:
\begin{equation}
\Omega\cap\mathcal{B}\subseteq\Omega_{0}
\end{equation}

\medskip{}

One may wonder if the reverse implication holds, i.e. are all $w\in\Omega_{0}$
balanced? The following proposition shows that this is not the case:
\begin{prop}
Assume (\ref{assumption:capacity}),(\ref{assumption:white}),(\ref{assumption:nonsingular}),
$2\le\min\{d_{x},d_{y}\},m$ and that $\sigma_{\max}\left(T\right)\neq0$,
then there exists an unbalanced flattest solution, namely: $\Omega_{0}\nsubseteq\Omega\cap\mathcal{B}$.
\end{prop}
\begin{proof}
First, notice that the set $\Omega_{0}\cap\mathcal{B}$ is non-empty.
Mulayoff et al. (2020) proposed the canonical flattest solution:
\begin{equation}
W_{m}^{\ast}=US_{m}^{\frac{1}{m}},\quad W_{j}^{\ast}=S_{j}^{\frac{1}{m}},\quad W_{1}^{\ast}=S_{1}^{\frac{1}{m}}V^{T}
\end{equation}

Where $S_{j}^{\frac{1}{m}}$ is used to denote the $d_{j}\times d_{j-1}$
matrix whose $k$th diagonal entry is $\left(\sigma_{k}\left(T\right)\right)^{\frac{1}{m}}$
and $\sigma_{k}\left(T\right)$ is the $k$th largest singular value
of $T$. By \thmref{sufficient} this is indeed a flattest solution,
and one can readily see that it is also balanced.

Now let $\left(\kaliseries W,m\right)\in\Omega_{0}\cap\mathcal{B}$
be a flattest balanced solution which satisfies the sufficient condition
in \thmref{sufficient}. Let $k\in\left[m-1\right]$ and denote the
SVD decomposition of $W_{k},\;W_{k+1}$ by: 
\[
W_{k}=U_{k}S_{k}V_{k}^{T},\;W_{k+1}=U_{k+1}S_{k+1}V_{k+1}^{T}
\]
Because $w$ is balanced we know that $V_{k+1}S_{k+1}^{2}V_{k+1}^{T}=U_{k}S_{k}^{2}U_{k}^{T}$
which implies:
\begin{equation}
U_{k}=V_{k+1},\quad S_{k}=S_{k+1}\label{eq:SvdChain}
\end{equation}

Without loss of generality, assume that $S_{k}$ is of the form $S_{k}=\begin{pmatrix}diag_{r}\left(\sigma_{1}^{(k)},\sigma_{2}^{(k)},\dots,\sigma_{r}^{(k)}\right)\\
\mathbb{0}_{\left(d_{k}-r\right)\times d_{k-1}}
\end{pmatrix}$ where\\
 $\left(\sigma_{k}\left(T\right)\right)^{\frac{1}{m}}=\sigma_{1}^{(k)}\ge\sigma_{2}^{(k)}\ge...\ge\sigma_{r}^{(k)}$.
We create a flattest unbalanced solution by ``perturbing'' $w$. 

Define $\bar{S}_{k+1}\in\mathbb{R}^{d_{k+1}\times\left(d_{k}+1\right)},\bar{S}_{k}\in\mathbb{R}^{\left(d_{k}+1\right)\times d_{k-1}}$,
$\bar{V}_{k+1},\bar{U}_{k}\in\mathbb{R}^{\left(d_{k}+1\right)\times\left(d_{k}+1\right)}$
by:
\[
\bar{S}_{k+1}=\begin{pmatrix}S_{k+1} & \mathbb{0}_{d_{k}\times1}\\
\mathbb{0}_{1\times d_{k}} & 0
\end{pmatrix},\ \bar{S}_{k}=\begin{pmatrix}S_{k} & \mathbb{0}_{d_{k}\times1}\\
\mathbb{0}_{1\times d_{k}} & \left(\sigma_{k}\left(T\right)\right)^{\frac{1}{m}}
\end{pmatrix},\ \bar{V}_{k+1}=\begin{pmatrix}V_{k+1} & \mathbb{0}_{d_{k}\times1}\\
\mathbb{0}_{1\times d_{k}} & 1
\end{pmatrix}=\bar{U}_{k}
\]

And $\bar{W}_{k+1}\in\mathbb{R}^{d_{k+1}\times\left(d_{k}+1\right)},\bar{W}_{k}\in\mathbb{R}^{\left(d_{k}+1\right)\times d_{k-1}}$
appropriately:
\[
\bar{W}_{k+1}=U_{k+1}\bar{S}_{k+1}\bar{V}_{k+1}^{T},\ \bar{W}_{k}=U_{k}\bar{S}_{k}V_{k}^{T}
\]

For $j\in\left[m\right]\backslash\left\{ k,k+1\right\} $set $\bar{W}_{j}=W_{j}$,
by (\ref{eq:SvdChain}) we have:

\begin{align*}
\bar{W}_{1:m} & =\prod_{j=k+2}^{m}W_{j}U_{k+1}\bar{S}_{k+1}{\color{gray}\underset{=I_{d_{k}}}{\underbrace{{\normalcolor \begin{pmatrix}V_{k+1}^{T} & \mathbb{0}_{d_{k}\times1}\\
\mathbb{0}_{1\times d_{k}} & 1
\end{pmatrix}\begin{pmatrix}V_{k+1} & \mathbb{0}_{d_{k}\times1}\\
\mathbb{0}_{1\times d_{k}} & 1
\end{pmatrix}}}}}\bar{S}_{k}V_{k}^{T}\prod_{j=1}^{k-1}W_{j}\\
 & =\prod_{j=k+2}^{m}W_{j}{\color{gray}\underset{\overset{(\ref{eq:SvdChain})}{=}W_{k+1}W_{k}}{\underbrace{{\normalcolor U_{k+1}\begin{pmatrix}S_{k+1} & \mathbb{0}_{d_{k}\times1}\\
\mathbb{0}_{1\times d_{k}} & 0
\end{pmatrix}\begin{pmatrix}S_{k} & \mathbb{0}_{d_{k}\times1}\\
\mathbb{0}_{1\times d_{k}} & \left(\sigma_{k}\left(T\right)\right)^{\frac{1}{m}}
\end{pmatrix}V_{k}^{T}}}}}W_{k+1}\prod_{j=1}^{k-1}W_{j}\\
 & =W_{1:m}=T\\
 & \Rightarrow\bar{W}_{1:m}\in\Omega
\end{align*}
\end{proof}
\begin{fact}
It remains to notice that by \thmref{sufficient} $\left(\kaliseries{\bar{W}},m\right)\in\Omega_{0}$,
and that the contra-positive of implication (\ref{eq:SvdChain}) yields
$\bar{w}$$\notin\mathcal{B}$.\newpage\printbibliography
\end{fact}

\end{document}
